\usetikzlibrary{positioning, chains, shapes.geometric, fit, shapes, arrows.meta, calc, backgrounds}

\begin{tikzpicture}[
    >=LaTeX, % Use default LaTeX arrows
    % Styles 
    node/.style={ % Input or output node
        circle,
        minimum width=2.25em,
        draw,
        fill=gray!10,
        thick,
        font=\scriptsize
    },
    group/.style={
        rectangle,
        rounded corners=1mm,
        draw,
        thick
    }, 
    innerNode/.style={
        circle,
        minimum width=0.3em,
        inner sep=0pt,
        draw,
        fill=gray!10,
        thick
    },
    arrow/.style={
        -latex,
        thick
    },
    backprop/.style={ % Backpropagation arrows
        arrow,
        dashed,
        gray
    }
]
    
    % Input
    \foreach \x in {1,...,2}
        \draw node at (0, -\x*1.25 - 0.625) [node] (first_\x) {$x_\x$};
    \draw node at (0, -5*1.25 + 0.625) [node] (first_n) {$x_n$};
    \path (first_2) -- (first_n) node[pos=0.5, scale=2] {$\vdots$};
    
    % Hidden 1
    \foreach \x in {1,...,2}
        \node at (2.5, -\x*1.25) [node] (second_\x){$\x$};
    \draw node at (2.5, -4*1.25) [node] (second_n) {$n$};
    \path (second_2) -- (second_n) node[pos=0.5, scale=2] {$\vdots$};
    \draw node at (2.5, -5*1.25) [node] (second_a1) {$a_1$};

    % Hidden 2
    \newcommand{\nodeGroup}[2]{
      \foreach \x in {1,...,2}
      \node at (5, #2-\x*0.25) [innerNode] (#1_\x){};
      \node at (5, #2-4*0.25) [innerNode] (#1_om){};
      \path (#1_2) -- (#1_om) node[pos=0.5, scale=0.5] {$\vdots$};
      \node[group, dashed, fit=(#1_1) (#1_om)] (#1) {};
    }
    \nodeGroup{third_1}{-0.25}
    \nodeGroup{third_kn}{-2.4}
    \path (third_1) -- (third_kn) node[pos=0.5, scale=2] {$\vdots$};

    \draw node at (5, -4.5) [node] (third_a2) {$a_2$};
    \draw node at (5, -4.5-1*1.1) [node] (third_c1) {$c_1$};
    \draw node at (5, -4.5-2*1.1) [node] (third_c2) {$c_2$};
    
    % Output
    \foreach \x in {1,...,2}
        \node at (7.5, -\x*1.25 - 0.625) [node] (fourth_\x){$y_\x$};
    \draw node at (7.5, -5*1.25 + 0.625) [node] (fourth_m) {$y_m$};
    \path (fourth_2) -- (fourth_m) node[pos=0.5, scale=2] {$\vdots$};

    % Input -> hidden 1
    \foreach \i in {1,2,n}
        \draw [arrow] (first_\i) to (second_\i);

    % A1 -> hidden 1
    \foreach \i in {1,2,n}
        \draw [arrow, bend left=40, color=gray] (second_a1) to (second_\i);

    % C1/C2/A2 -> hidden 2
    \foreach \i in {c1,c2,a2}
        \foreach \j in {1,kn}
            \draw [arrow, bend left=40, color=gray] (third_\i) to (third_\j);
    
    % hidden 1 -> hidden 2
    \foreach \i in {1,2,n}
        \foreach \j in {1,kn}
        \draw [arrow] (second_\i) to (third_\j);

    \draw [arrow, bend left=20] (third_kn) to (third_1);

    % hidden 2 -> Output
    \foreach \i in {1,kn}
        \foreach \j in {1,2,m}
        \draw [arrow] (third_\i) to (fourth_\j);

    % loops
    \path[arrow, color=gray] (second_a1) edge[loop left] ();
    \path[arrow, color=gray] (third_a2) edge[loop left] ();
    \path[arrow, color=gray] (third_c1) edge[loop left] ();
    \path[arrow, color=gray] (third_c2) edge[loop left] ();
    \path[arrow] (third_1) edge[loop left] ();
    \path[arrow] (third_kn) edge[loop left] ();

    % Layer Labels
    \node[above=0.25cm of first_1] (xlabel) {Input};
    \node[above=0.25cm of second_1] (h1label) {First hidden layer};
    \node[above=0.25cm of third_1] (h2label) {Second hidden layer};
    \node[above=0.25cm of fourth_1] (ylabel) {Output};
\end{tikzpicture}

% adapted from https://github.com/fraserlove/nntikz/tree/main
