\usetikzlibrary{positioning, chains, shapes.geometric, fit, shapes, arrows.meta, calc, backgrounds}

\begin{tikzpicture}[
    >=LaTeX, % Use default LaTeX arrows
    % Styles 
    node/.style={ % Input or output node
        circle,
        minimum width=2.25em,
        inner sep=0.5pt,
        draw,
        fill=gray!10,
        thick,
        font=\scriptsize
    },
    group/.style={
        rectangle,
        rounded corners=1mm,
        draw,
        thick
    }, 
    innerNode/.style={
        circle,
        minimum width=0.3em,
        inner sep=0pt,
        draw,
        fill=gray!10,
        thick
    },
    arrow/.style={
        -latex,
        thick
    },
    backprop/.style={ % Backpropagation arrows
        arrow,
        dashed,
        gray
    }
]
    
    % Input
    \foreach \x in {1,...,1}
        \draw node at (0, -\x*1.25 - 1.25) [node] (first_\x) {$x_\x$};
    \draw node at (0, -3*1.25 -1.25) [node] (first_n) {$x_n$};
    \path (first_1) -- (first_n) node[pos=0.5, scale=2] {$\vdots$};
    
    % Hidden 1
    \foreach \x in {1,...,1}
        \node at (2.5, -0.625-\x*1.25) [node] (second_\x){$\x$};
    \draw node at (2.5, -0.625-3*1.25) [node] (second_n) {$n$};
    \path (second_1) -- (second_n) node[pos=0.5, scale=2] {$\vdots$};
    \draw node at (2.5, -0.625-4*1.25) [node] (second_a1) {$a_1$};

    % Hidden 2
    \node at (5, 0.25-1.1) [node] (third_1_1){$ι_j(1)$};
    \node at (5, 0.25-2*1.1-0.75) [node] (third_1_n){$ι_j(n)$};
    \path (third_1_1) -- (third_1_n) node[pos=0.5, scale=2] {$\vdots$};
    \node at (5, 0.25-3*1.1-0.75) [node] (third_1_om){$ω_j$};
    \node[group, dashed, fit=(third_1_1) (third_1_om)] (third_1) {};

    \draw node at (5, -6) [node] (third_a2) {$a_2$};
    \draw node at (5, -6-1*1.1) [node] (third_c1) {$c_1$};
    \draw node at (5, -6-2*1.1) [node] (third_c2) {$c_2$};

    \path (third_1) -- (third_a2) node[pos=0.5, scale=2] (third_vdots) {$\vdots$};
    
    % Output
    \foreach \x in {1,...,2}
        \node at (7.5, -\x*1.25 - 0.625) [node] (fourth_\x){$y_\x$};
    \draw node at (7.5, -5*1.25 + 0.625) [node] (fourth_m) {$y_m$};
    \path (fourth_2) -- (fourth_m) node[pos=0.5, scale=2] {$\vdots$};

    % Input -> hidden 1
    \foreach \i in {1,n}
        \draw [arrow] (first_\i) to (second_\i);

    % A1 -> hidden 1
    \foreach \i in {1,n}
        \draw [arrow, bend left=40, color=gray] (second_a1) to (second_\i);

    % C1/C2/A2 -> hidden 2
    \foreach \i in {c1,c2}
        \foreach \j in {1,n}
            \draw [arrow, bend left=40, color=gray] (third_\i) to (third_1_\j);
    \draw [arrow, bend left=40, color=gray] (third_a2) to (third_1_om);

    % connect group to itself
    \foreach \i in {1,n}
    \foreach \j in {om} {
      \draw [arrow, bend left=60] (third_1_\j) to (third_1_\i);
      \draw [arrow, bend right=30] (third_1_\i) to (third_1_\j);
    }

    % group to other groups
    \draw [arrow, bend left=30] (4.9, -4.9) to (third_1_om);
    \draw [arrow, bend right=20] (third_1_om) to (4.95, -4.7);
    
    % hidden 1 -> hidden 2
    \foreach \i in {1,n}
        \draw [arrow] (second_\i) to (third_1_\i);

    % hidden 2 -> Output
    \foreach \i in {1}
        \foreach \j in {1,2,m}
        \draw [arrow] (third_\i) to (fourth_\j);

    % loops
    \path[arrow, color=gray] (second_a1) edge[loop left] ();
    \path[arrow, color=gray] (third_a2) edge[loop left] ();
    \path[arrow, color=gray] (third_c1) edge[loop left] ();
    \path[arrow, color=gray] (third_c2) edge[loop left] ();
    \path[arrow] (third_1_om) edge[loop left] ();

    % Layer Labels
    \node[above=0.25cm of first_1] (xlabel) {Input};
    \node[above=0.25cm of second_1] (h1label) {First hidden layer};
    \node[above=0.25cm of third_1] (h2label) {Second hidden layer};
    \node[above=0.25cm of fourth_1] (ylabel) {Output};
\end{tikzpicture}

% adapted from https://github.com/fraserlove/nntikz/tree/main
