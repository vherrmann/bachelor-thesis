\tikzstyle{descript} = [text = black,align=center, minimum height=1.8cm, align=center, outer sep=0pt,font = \footnotesize]
\tikzstyle{activity} =[align=center,outer sep=1pt]
\definecolor{ColorOne}{named}{MidnightBlue}

% #1 = color
% #2 = left coordinate
% #3 = right coordinate
\newcommand{\FillRegion}[3]{%
  \fill[#1] (#2) -- (#3) -- ($(#3)+(0,1)$) -- ($(#2)+(0,1)$);
}

\newenvironment{timeline}{
\begin{tikzpicture}[very thick, black]
  \small
  
  \coordinate (O) at (0,0);
  \coordinate (T2-1) at (3,0);
  \coordinate (T2) at (5,0);
  \coordinate (T3) at (7,0);
  \coordinate (T4) at (9,0);
  \coordinate (T4+1) at (11,0);
  \coordinate (T) at (14,0);
  
  \FillRegion{ColorOne!10!white}{O}{T2-1}

  \fill[
    pattern={Lines[angle=45,distance=6pt,line width=3pt, xshift=4.5pt]},
    pattern color=ColorOne!20!white
  ] (T2-1) rectangle ($(T2)+(0,1)$);
  \fill[
    pattern={Lines[angle=45,distance=6pt,line width=3pt]},
    pattern color=ColorOne!10!white
  ] (T2-1) rectangle ($(T2)+(0,1)$);

  \FillRegion{ColorOne!30!white}{T2}{T3}
  \FillRegion{ColorOne!40!white}{T3}{T4}
  \FillRegion{ColorOne!50!white}{T4}{T}
  % \FillRegion{ColorOne!60!white}{T4+1}{T}
  
  \draw ($(O)!0.5!(T2-1)+(0,0.5)$) node[activity] {$T_1$};
  \draw ($(T2-1)!0.5!(T2)+(0,0.5)$) node[activity] {$T_2$};
  \draw ($(T2)!0.5!(T3)+(0,0.5)$) node[activity] {$T_3$};
  \draw ($(T3)!0.5!(T4)+(0,0.5)$) node[activity] {$T_4$};
  \draw ($(T4)!0.5!(T)+(0,0.5)$) node[activity] {$T_5$};

  %% Timeline
  \draw[dotted] (O) -- (T2-1);
  \draw (T2-1) -- (T4+1);
  \draw[dotted] (T4+1) -- (T);
  \draw[->] (T) -- ($(T)+(0.5,0)$);

  %% Ticks
  \foreach \pos/\labeltext in {
    O/$0$,
    T2-1/$T_2-1$,
    T2/$T_2$,
    T3/$T_3$,
    T4/$T_4$,
    T4+1/$T_4+1$,
    T/$T$
  }{
    % Draw tick
    \draw (\pos) ++(0,3pt) -- ++(0,-6pt);
    % Add label below
    \node[below=4pt, anchor=north] at (\pos) {\labeltext};
  }
}{\end{tikzpicture}}

% adapted from https://www.overleaf.com/project/68b326ec3dceb466ee45cc28

% #1 = coordinate
% #2 = label
\NewDocumentCommand{\SpikeOut}{O{} O{} O{} m}{%
        \draw[->,thick,color=black,#1] ($(#4)+(0,0)$) -- ($(#4)+(0,1.5)$) node [above=0pt,align=center,black] {#2} node[midway, sloped, yshift=7pt] {\color{gray} #3};
}
\NewDocumentCommand{\SpikeIn}{O{} O{} O{} m}{%
        \draw[<-,thick,color=black,#1] ($(#4)+(0,0)$) -- ($(#4)+(0,1.5)$) node [above=0pt,align=center,black]  {#2} node[pos=0,anchor=north] {\rotatebox{90}{#3}};
}

% #1 = number of holes/number of points-1
% #2 = size of circles
% #3 = coordinate 1
% #4 = coordinate 2
\newcommand{\dotsT}[4]{
  \foreach \i in {0,...,#1} {
    \coordinate (P\i) at ($ (#3)!{\i/#1}!(#4) $);
    \fill (P\i) circle (#2); % draw point
  }
}

% https://tex.stackexchange.com/questions/604337/comparing-node-coordinates-in-tikz
\makeatletter
\newcommand{\gettikzxy}[3]{%
  \tikz@scan@one@point\pgfutil@firstofone#1\relax
  \edef#2{\the\pgf@x}%
  \edef#3{\the\pgf@y}%
}
\makeatother

% #1 = coordinate 1
% #2 = coordinate 2
\newcommand{\dotsBtSp}[3][15]{
  \dotsT{#1}{0.5pt}{$(#2)+(0,1.3)$}{$(#3)+(0,1.3)$};
  \dotsT{#1}{0.5pt}{$(#2)+(0,0.2)$}{$(#3)+(0,0.2)$};
}

\newcommand{\drawSpikeOutFrom}[1]{
  \gettikzxy{(#1)}{\px}{\py}
  \foreach \name in {T2,T3,T4,T4+1} {
    \gettikzxy{(\name)}{\qx}{\qy}

    \ifdimcomp{\qx}{>=}{\px}{
        \SpikeOut{$(\name)-(0.3,0)$}{};
    }{}
  }
  \dotsBtSp{$(T4+1)-(0.1,0)$}{$(T)+(-0.5,0)$};
  \SpikeOut{$(T)+(-0.3,0)$}{};
}

\NewDocumentCommand{\SpikeOutAt}{O{} O{} O{} m}{%
  \SpikeOut[#1][#2][#3]{$(#4)-(0.3,0)$};
}

