\section{Definitions}
\label{ch:defs}

% TODO: graphic for whole layer
Our type of SNN should be thought of as a composition of an initial input layer, a number of hidden spiking layers with internal state and an affine-linear layer mapping spikes activations over time, so called spike trains, to the value of the output layer.
Like motivated we are adding additional structure to the definitions~\cite{nguyen2025timespikeunderstandingrepresentational}.
We shall first define the input \(i^{[l]}(t)\), the membrane potential before spike \(p^{[l]}(t)\), the membrane potential after spike \(u^{[l]}(t)\) and the spike activations \(s^{[l]}(t)\) of the hidden layers:

% TODO: graphic for single neuron/single layer
%TODO: herleitung: ~/Downloads/1901.09948v2.pdf
\begin{definition}
  The \textbf{input vector} \(i^{[l]}(t)∈\{0,1\}^{n_l}\), the \textbf{spike vector} \(s^{[l]}(t)∈\{0,1\}^{n_l}\) and the \textbf{membrane potential vector} \(u^{[l]}(t)\) of a hidden layer \(λ=(W^{[l]},b^{[l]},u^{[l]}(0),i^{[l]}(0),α^{[l]},β^{[l]},ϑ^{[l]})\), \(l∈[L]\), are recursively defined as
  \begin{align}
    i^{[l]}(t) & ≔ α^{[l]}i^{[l]}(t-1)+W^{[l]}s^{{(l)}-1}(t)+V^{[l]}s^{[l]}(t-1) \\
    p^{[l]}(t) & ≔ β^{[l]}u^{[l]}(t-1)+i^{[l]}(t)+b^{[l]} \\
    s^{[l]}(t) & ≔ H(p^{[l]}(t)-ϑ1_{n_l}) \\
    u^{[l]}(t) & ≔ p^{[l]}(t)-ϑs^{[l]}(t)
  \end{align}
  with \(s^{[l]}(0)=0\) and given
  \begin{enumerate}
    \item[•] \textbf{initial membrane potential}: \(u^{[l]}(0)∈ℝ^{n_l}\),
    \item[•] \textbf{initial input}: \(i^{[l]}(0)∈ℝ^{n_l}\),
    \item[•] \textbf{weight matrices}: \(W^{[l]}∈ℝ^{n_l×n_{l-1}}\), \(V^{[l]}∈ℝ^{n_l×n_l}\),
    \item[•] \textbf{bias vectors}: \(b^{[l]}∈ℝ^{n_l}\),
    \item[•] \textbf{leaky terms}: \(α^{[l]},β^{[l]}∈[0,1]\),
    \item[•] \textbf{threshold}: \(ϑ^{[l]}∈(0,∞)\),
  \end{enumerate}
  where \(H≔𝟙_{[0,∞)}\) is a step function and \(T∈ℕ\) is the number of simulated time steps.
\end{definition}

We further define recurrent d.t. LIF-SNN and the function the network realizes:
\begin{definition}
  A \textbf{recurrent discrete-time LIF-SNN} of \textbf{depth} \(L\) with \textbf{layer-widths} \((n_0,…,n_{L+1})\) and \(T∈ℕ\) time-steps is given by
  \[ Φ≔((W^{[l]},b^{[l]},V^{[l]},u^{[l]}(0),i^{[l]}(0),α^{[l]},β^{[l]},ϑ^{[l]})_{l∈[L]},T,(E,D)) \]
  where the \textbf{input encoder} \(E:ℝ^{n_0}→ℝ^{n_0×T}\) maps a vector \(x∈ℝ^{n_0}\) to a corresponding first layer spike activation \(s^{[0]}=E(x)\) and the \textbf{output decoder} \(D:\{0,1\}^{n_L×T}→ℝ^{n_{L+1}}\) maps the spike activations of the last hidden layer to real values.
\end{definition}

\begin{definition}
  A recurrent discrete-time LIF-SNN \(ϕ\) \textbf{realizes} the function \(R(Φ):ℝ^{n_0}→ℝ^{n_{L+1}}\):
  \[ R(Φ)(x)=D((s^{[L]}(t))_{t∈[T]})\quad \text{with }s^{[0]}≔E(x)\]
\end{definition}

\begin{definition}
  A recurrent discrete-time LIF-SNN employs \textbf{direct encoding} if we have
  \[ ∀_{t∈[T]}E(x)(t)=x \]
  for the input encoder and has \textbf{membrane potential outputs} if the output decoder can be written as
  \[ D((s(t))_{t∈[T]})=\sum_{t=1}^Ta_t(W^{L+1}s(t)+b^{L+1}) \]
  for some \((a_t)_{t∈[T]}∈ℝ^T\), \(b^{L+1}∈ℝ^{n_{L+1}}\) and \(W^{L+1}∈ℝ^{n_{L+1}×n_L}\).
\end{definition}
We will only consider recurrent discrete-time LIF-SNN with direct encoding and membrane potential outputs. In fact, we will use “recurrent discrete-time LIF-SNN” to mean “\dtlifsnn with direct encoding and membrane potential”.

For clearer construction of our networks we will additionally define neurons:

% TODO: just notation?
\begin{definition}
  The \(i\)-th neuron of a hidden layer \(λ=(W^{[l]},b^{[l]},V^{[l]},u^{[l]}(0),i^{[l]}(0),α^{[l]},β^{[l]},ϑ^{[l]})\) of a \dtlifsnn is a tuple \((w,b,v,u_0,i_0)\) with \(w∈ℝ^{n_{l-1}}\), \(v∈ℝ^{n_l}\) and \(b,u_0,i_0∈ℝ\), such that \(b\), \(u_0\), \(i_0\) are the \(i\)-th component of \(b^{[l]}\), \(u^{[l]}(0)\), \(i^{[l]}(0)\) respectively and \(w\), \(v\) are the \(i\)-th row vector of \(W^{[l]}\), \(V^{[l]}\) respectively.
\end{definition}

The following lemma will be very helpful for the proofs in the following sections:
\begin{lemma}\label{lem:non-recursive-defs}
  We define the following non-recursive formulas with explicit reference to the previous spikes. Let \(1≤t≤T\) and \((s_p^{[l]})_{l∈\{0…L'\}}\) such that \(s_p^{[l]}:\{0,1\}^{n_l×t'_l}\) for \(l∈\{0…L'\}\). \(L'\) and \(∀_lt'_l\) have to be chosen such that the following terms are well-defined, e.g. \(l'=L\) and \(∀_lt'_l=T\)
  \begin{align}
   i^{[l]}(t;(s_p^{[l]})_l)&≔(α^{[l]})^ti^{[l]}(0)+\sum_{k=1}^t(α^{[l]})^{t-k}\left(W^{[l]}s_p^{[l-1]}(k)+V^{[l]}s_p^{[l]}(k-1)\right), \\
   p^{[l]}(t;(s_p^{[l]})_l) &≔ (β^{[l]})^tu^{[l]}(0)+\sum_{k=1}^t(β^{[l]})^{t-k}\left(i^{[l]}(k;(s_p^{[l]})_l)+b^{[l]}\right)-ϑ\sum_{k=1}^{t-1}(β^{[l]})^{t-k}s_p^{[l]}(k),\\
   s^{[l]}(t;(s_p^{[l]})_l) & ≔ H(p^{[l]}(t;(s_p^{[l]})_l)-ϑ1_{n_l}) \\
   u^{[l]}(t;(s_p^{[l]})_l) &≔ (β^{[l]})^tu^{[l]}(0)+\sum_{k=1}^t(β^{[l]})^{t-k}\left(i^{[l]}(k;s_p)+b^{[l]}-ϑs_p^{[l]}(k)\right),
  \end{align}
  These formulas are equivalent to the previous recursive definitions, given \(s_p^{[l]}=s^{[l]}\) for \(l∈\{0…L'\}\).
\end{lemma}

\begin{proof}
  Let the time \(1≤t≤T\) be non-zero. By induction we immediately get
  \[ i^{[l]}(t)=(α^{[l]})^ti^{[l]}(0)+\sum_{i=1}^t(α^{[l]})^{t-i}\left(W^{[l]}s^{[l-1]}(i)+V^{[l]}s^{[l]}(i-1)\right) \]
  By definition of \(u^{[l]}\) and \(p^{[l]}\) we further obtain
  \[ u^{[l]}(t) = β^{[l]}u^{[l]}(t-1)+i^{[l]}(t)+b^{[l]}-ϑs^{[l]}(t) \]
  from which
  \[ u^{[l]}(t) = (β^{[l]})^tu^{[l]}(0)+\sum_{i=1}^t(β^{[l]})^{t-i}\left(i^{[l]}(i)+b^{[l]}-ϑs^{[l]}(i)\right) \]
  follows by induction.
  By substituting \(u^{[l]}\) in \(p^{[l]}\) we further obtain:
  \[ p^{[l]}(t) = (β^{[l]})^tu^{[l]}(0)+\sum_{i=1}^t(β^{[l]})^{t-i}\left(i^{[l]}(i)+b^{[l]}\right)-ϑ\sum_{i=1}^{t-1}(β^{[l]})^{t-i}s^{[l]}(i) \]
\end{proof}

Let us now take a look at some simple examples:

%TODO: examples überarbeiten
\begin{example}
  Let \(T,L∈ℕ\). Then there exists a d.t. LIF-SNN with \(∀_{t∈[T]}s^{[L]}(t)=s^{[0]}(t)\) for any \(s^{[0]}∈\{0,1\}^{n_0×T}\).

  We can proof this by using constant width \(n_l=n\), weights \(W^{[l]}=I_n\), biases \(b^{[l]}=0\), initial membrane potential \(u^{[l]}(0)=0\), leaky term \(β^{[l]}=0\) and threshold \(ϑ^{[l]}=1\) for all \(l∈[L]\).

  We then get by definition:
  \begin{alignat*}{3}
    s^{[l]}(t) & = H(s^{[l-1]}(t)-1_{n_l})  &= \ s^{[l-1]}(t) \\
    u^{[l]}(t) & = \phantom{H(}s^{[l-1]}(t)-s^{[l]}(t)&= 0
  \end{alignat*}
\end{example}

\begin{example}\label{ex:2}
  Let \(T,L∈ℕ\). Then there exists a d.t. LIF-SNN with \(∀_{t∈T}s^{[L]}(t)=\max_{t'∈[t-1]}(s^{[0]}(t'))\) for any \(s^{[0]}∈\{0,1\}^{n_0×T}\): In this network an output neuron switches on when the corresponding input neurons fires and does not switch off later.

  We can proof this by using constant width \(n_l=n\), weights \(W^{[l]}=T·I_n\), biases \(b^{[l]}=0\), initial membrane potential \(u^{[l]}(0)=0\), leaky term \(β^{[l]}=1\) and threshold \(ϑ^{[l]}=1\) for all \(l∈[L]\).

  We then get by definition:
  \begin{alignat*}{2}
    s^{[l]}(t) & = H(&u^{[l]}(t-1)+T·s^{[l-1]}(t)-1_{n_l}) \\
    u^{[l]}(t) & = &u^{[l]}(t-1)+T·s^{[l-1]}(t)-s^{[l]}(t)
  \end{alignat*}
  By adding over all timesteps we obtain
  \begin{alignat*}{1}
    s^{[l]}(t) & = H(T\sum_{i=1}^ts^{[l-1]}(i)-(\sum_{t=i}^{t-1}s^{[l]}(i))+1_{n_l}) ) \\
    u^{[l]}(t) & = \sum_{i=1}^t(T·s^{[l-1]}(i)-s^{[l]}(i))
  \end{alignat*}
  Since \(\sum_{i=1}^{t-1}s^{[l]}(i)+1_{n_l}≤T\), once there is any \(t_0∈[T]\) with \(s^{[l-1]}_i(t_0)=1\) for an \(i\), we get \(∀_{t≥t_0}s^{[l]}_i(t)=1\).
  By induction over the layers we clearly get the required property.
\end{example}
