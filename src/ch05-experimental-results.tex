\section{Experimental results} \label{ch:experiments}

To try to better understand the landscape of the input of a \rdtlifsnn to proof~\cref{thm:bound-regions}, we have created a few programs.

\subsection{Computing the number of regions}

We assume \(W=I_n\) yet again, since we the algorithms become much simpler and more efficient. Also once we understand the situation for \(W=I_n\), we are hopeful to proof it for arbitrary \(W\).

Let further \(g\) in~\cref{alg:compute_regions} be defined as in~\cref{lem:invert-p}.

%TODO: add explanation
\begin{algorithm}[H]
\caption{Recursive Region Computation}
\label{alg:compute_regions}
\begin{algorithmic}
\Function{ComputeRegionsStartingWithST}{$t, st, x^C, y^C$}
    \If{$t > T$}
        \State \Return $\{(st, x^C, y^C)\} $
    \EndIf
    \State $x \gets g(t;st) $
    \State $ subRegions \gets \{ [x',y') \mid x'<y',\ ∀_ix'_i∈\{x^C_i,x_i\},\ ∀_iy'_i∈\{y^C_i,x_i\} \} $
    \For{$ C ∈ subRegions $}
      \State $ newST \gets \operatorname{append}(st,[(\mathbf{1}_{\{x_i≤x'_i\}})_{i∈[n_1]}]) $
      \State \Call{ComputeRegionsStartingWithST}{$t+1,newST,x',y'$}
    \EndFor
\EndFunction

%TODO: implement α stuff

\Function{ComputeRegions}{$ $}

    \State \Call{ComputeRegionsStartingWithST}{$1,[(0,…,0)],(-∞,…,-∞),(∞,…,∞)$}
\EndFunction

\end{algorithmic}
\end{algorithm}

%TODO: analysis

% TODO: abstract algorithms
% TODO: images
% TODO: numerical considerations
